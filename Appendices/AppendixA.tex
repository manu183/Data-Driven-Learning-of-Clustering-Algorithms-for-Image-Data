\chapter{The Hungarian Method}
\label{sec:hungarian}

Our goal is to find the best possible matching between two clusterings $C_1^i, ..., C_k^i$ and $C_1^j, ..., C_k^j$. In order to do so, we calculate the cost of matching each possible pair of clusters within the two clusterings. 
% Table \ref{app:hung:distances} shows how such a cost matrix can look like.

% \begin{table}[h]
%     \centering
%     \begin{tabular}{|l | l l l l l|}
%     \hline
%     j\textbackslash i & 1 & 2 & 3 & 4 & 5\\ \hline
%     1 & 20 & 15 & 30 & 50 & 40\\
%     2 & 80 & 10 & 15 & 20 & 30\\
%     3 & 20 & 30 & 50 & 80 & 60\\
%     4 & 30 & 50 & 40 & 20 & 10\\
%     5 & 20 & 30 & 40 & 50 & 25\\ \hline
%     \end{tabular}
%     \caption{In order to determine the optimal matching between two clusterings, we note the pairwise matching costs.}
%     \label{app:hung:distances}
% \end{table}

To find the optimal matching in a brute force way, we have to look at each possible matching. Say we want to match each $i$ to one $j$. For $i = 1$ we can pick from 5 different values of $j$, for $i = 2$ there are 4 potential values of $j$. This will overall result in $k! = 5! = 120$ different combinations, thus the complexity of the brute force approach is $O(k!)$. A more efficient algorithm (especially for higher values of $k$) was introduced by Kuhn and Munkres \cite{kuhn1955hungarian}\cite{munkres1957algorithms}. It consists of three major steps. In the first one, we subtract the row minima from each row. This step is performed in table \ref{app:hung:step1}.

\begin{table}[h]
    \centering
    \begin{tabular}{|l | l l l l l| l |}
    \hline
    j\textbackslash i & 1 & 2 & 3 & 4 & 5 & \\ \hline
    1 & 5 & 0 & 15 & 35 & 25 & (-15)\\
    2 & 70 & 0 & 5 & 10 & 20 & (-10)\\
    3 & 0 & 10 & 30 & 60 & 40 & (-20)\\
    4 & 20 & 40 & 30 & 10 & 0 & (-10)\\
    5 & 0 & 10 & 20 & 30 & 5 & (-20)\\ \hline
    \end{tabular}
    \caption{Hungarian method step 1: Subtract the row minima from each row.}
    \label{app:hung:step1}
\end{table}

After subtracting the row minima, we now also subtract the column minima from each column as shown in table \ref{app:hung:step2}.

\begin{table}[h]
    \centering
    \begin{tabular}{|l | l l l l l |}
    \hline
    j\textbackslash i & 1 & 2 & 3 & 4 & 5\\ \hline
    1 & 5 & 0 & 10 & 25 & 25\\
    2 & 70 & 0 & 0 & 0 & 20\\
    3 & 0 & 10 & 25 & 50 & 40\\
    4 & 20 & 40 & 25 & 0 & 0\\
    5 & 0 & 10 & 15 & 20 & 5\\ \hline
    & - & - & (-5) & (-10) & -\\ \hline
    \end{tabular}
    \caption{Hungarian method step 2: Subtract the column minima from each column.}
    \label{app:hung:step2}
\end{table}

Now we try to find the optimal matching. To do so, we cover all zeros with lines and count the minumum needed lines to do so. Table \ref{app:hung:step3} shows that we need four lines.

\begin{table}[h]
    \centering
    \begin{tabular}{|l | l l l l l |}
    \hline
    j\textbackslash i & 1 & 2 & 3 & 4 & 5\\ \hline
    1 & \cellcolor{orange!25}5 & \cellcolor{orange!50}0 & 10 & 25 & 25\\
    2 & \cellcolor{orange!75}70 & \cellcolor{orange!75}0 & \cellcolor{orange!75}0 & \cellcolor{orange!75}0 & \cellcolor{orange!75}20\\
    3 & \cellcolor{orange!25}0 & \cellcolor{orange!50}10 & 25 & 50 & 40\\
    4 & \cellcolor{orange!75}20 & \cellcolor{orange!75}40 & \cellcolor{orange!75}25 & \cellcolor{orange!75}0 & \cellcolor{orange!75}0\\
    5 & \cellcolor{orange!25}0 & \cellcolor{orange!50}10 & 15 & 20 & 5\\ \hline
    \end{tabular}
    \caption{Hungarian method step 3: Cover all zeros with as few lines as possible.}
    \label{app:hung:step3}
\end{table}

After covering the zeros and counting the lines, we found the optimal matching in case the number of lines equals the number of rows (or columns) in the matrix. As we need four lines and the matrix has five rows in this example, we have to add more zeros. To do that, we subtract the minimum value of the matrix (which is 5 here) from all uncovered values that are not zero and add it to all values that are not zero and covered twice. Now we can again check the needed lines as in table \ref{app:hung:additionalstep}.

\begin{table}[h]
    \centering
    \begin{tabular}{|l | l l l l l |}
    \hline
    j\textbackslash i & 1 & 2 & 3 & 4 & 5\\ \hline
    1 & 5 & 0 & 5 & 20 & 20\\
    2 & 75 & 0 & 0 & 0 & 20\\
    3 & 0 & 10 & 20 & 45 & 35\\
    4 & 25 & 45 & 25 & 0 & 0\\
    5 & 0 & 10 & 10 & 15 & 0\\ \hline
    \end{tabular}
    \begin{tabular}{|l | l l l l l |}
        \hline
        j\textbackslash i & 1 & 2 & 3 & 4 & 5\\ \hline
        1 & \cellcolor{orange!25}5 & \cellcolor{orange!50}0 & 5 & 20 & 20\\
        2 & \cellcolor{orange!75}75 & \cellcolor{orange!75}0 & \cellcolor{orange!75}0 & \cellcolor{orange!75}0 & \cellcolor{orange!75}20\\
        3 & \cellcolor{orange!25}0 & \cellcolor{orange!50}10 & 20 & 45 & 35\\
        4 & \cellcolor{orange!75}25 & \cellcolor{orange!75}45 & \cellcolor{orange!75}25 & \cellcolor{orange!75}0 & \cellcolor{orange!75}0\\
        5 & \cellcolor{orange!100}0 & \cellcolor{orange!100}10 & \cellcolor{orange!100}10 & \cellcolor{orange!100}15 & \cellcolor{orange!100}0\\ \hline
    \end{tabular}
    \caption{Hungarian method additional step: Create more zeroes until the number of minimal needed lines to cover all zeros matches the number of rows.}
    \label{app:hung:additionalstep}
\end{table}

This will then result in the assignment seen in table \ref{app:hung:assignment}. Applying the matching to the input matrix then gives the optimal cost by summing the optimal values. For this example the optimal cost is then 95.

\begin{table}[h]
    \centering
    \begin{tabular}{|l | l l l l l |}
    \hline
    j\textbackslash i & 1 & 2 & 3 & 4 & 5\\ \hline
    1 & 5 & \cellcolor{blue!25}0 & 5 & 20 & 20\\
    2 & 75 & 0 & \cellcolor{blue!25}0 & 0 & 20\\
    3 & \cellcolor{blue!25}0 & 10 & 20 & 45 & 35\\
    4 & 25 & 45 & 25 & \cellcolor{blue!25}0 & 0\\
    5 & 0 & 10 & 10 & 15 & \cellcolor{blue!25}0\\ \hline
    \end{tabular}
    \begin{tabular}{|l | l l l l l|}
        \hline
        j\textbackslash i & 1 & 2 & 3 & 4 & 5\\ \hline
        1 & 20 & \cellcolor{blue!25}15 & 30 & 50 & 40\\
        2 & 80 & 10 & \cellcolor{blue!25}15 & 20 & 30\\
        3 & \cellcolor{blue!25}20 & 30 & 50 & 80 & 60\\
        4 & 30 & 50 & 40 & \cellcolor{blue!25}20 & 10\\
        5 & 20 & 30 & 40 & 50 & \cellcolor{blue!25}25\\ \hline
    \end{tabular}
    \caption{Result of the Hungarian method: The optimal matching between two clusterings.}
    \label{app:hung:assignment}
\end{table}