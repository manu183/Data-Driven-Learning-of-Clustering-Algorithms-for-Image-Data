\begin{abstract}
Unüberwachtes Lernen ist in vielen Bereichen sehr populär, bspw. wird es verwendet, um Bilder oder Webseiten in ähnliche Gruppen einzuteilen, um Ausreißer in Datensätzen zu erkennen oder um Nutzerempfehlungen zu generieren. Dabei gibt es viele verschiedene Algorithmen, deren Ergebnisse sich oft unterscheiden. In Abhängigkeit der Daten eignen sich Algorithmen mehr oder weniger gut für verschiedene Aufgaben, weshalb es meist nicht trivial ist, den besten Algorithmus für eine gegebene Aufgabe zu bestimmen. Da oft keine genaueren Informationen über den verwendeten Datensatz zugrundeliegen, müssen konservative Annahmen für die Auswahl der Algorithmen getroffen werden. In dieser Arbeit wird das Problem der Auswahl des passenden Algorithmus für agglomeratives hierarchisches Clustering durch eine parametrisierte Distanzfunktion gelöst, die es ermöglicht, linear zwischen verschiedenen Algorithmen zu interpolieren. Dabei werden mit Single Linkage, Average Linkage und Complete Linkage drei mögliche Algorithmen zur Berechnung der Distanz zweier Cluster berücksichtigt.\\

Das in dieser Arbeit vorgestellte Verfahren findet garantiert den besten Algorithmus und liefert zusätzlich noch bessere Clusterings als alle verwendeten Linkage-Strategien durch eine gewichtete Linearkombination dieser. Die Ergebnisse und das Potential dieses Algorithmus werden mit verschiedenen Bild- und Textdaten aufgezeigt. Zusätzlich wird der vorgestellte Algorithmus auch dazu angewandt, die optimale Kombination verschiedener Metriken für Clusteringaufgaben zu finden, was ebenfalls zu deutlichen Verbesserungen der Ergebnisse führt.
\end{abstract}