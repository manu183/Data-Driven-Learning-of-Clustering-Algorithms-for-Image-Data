\begin{abstract}
Unüberwachtes Lernen ist in vielen Bereichen sehr populär, bspw. wird es verwendet um Bilder oder Webseiten in ähnliche Bereiche einzuteilen, um Ausreißer in Datensätzen zu erkennen oder um Nutzern Empfehlungen zu geben. Dabei gibt es viele verschiedenen Algorithmen, die zu verschiedenen Ergebnissen führen. Abhängig von den Daten eignen sich Algorithmen mehr oder weniger für verschiedene Aufgaben, weshalb es meist auch nicht trivial ist, welcher Algorithmus der beste für eine gegebene Aufgabe ist. Da oft keine genaueren Informationen über den verwendeten Datensatz zugrundeliegen, können nur schlechtmöglichste Annahmen für die Auswahl von Algorithmen berücksichtigt werden. In dieser Arbeit wird das Problem der Auswahl des passenden Algorithmus für agglomeratives hierarchisches Clustering durch eine parametrisierte Distanz-Funktion gelöst, die es ermöglicht linear zwischen verschiedenen Algorithmen zu implementieren. Dabei werden mit Single Linkage, Average Linkage und Complete Linkage drei mögliche Algorithmen zur Berechnung der Distanz zweier Cluster berücksichtigt.\\

Der in dieser Arbeit vorgestellte Algorithmus findet garantiert den besten Algorithmus, liefert zusätzlich aber noch bessere Clusterings als alle verwendenten Linkage-Strategien durch eine gewichtete Linearkombination dieser. Die Ergebnisse und das Potential dieses Algorithmus werden mit verschiedenen Bild- und Textdaten aufgezeigt. Zusätlich wird der vorgestellte Algorithmus auch dazu angewandt, die optimale Kombination verschiedener Metriken für Clusteringaufgaben zu finden, was ebenfalls zu deutlichen Verbesserung der Ergebnisse führt.
\end{abstract}