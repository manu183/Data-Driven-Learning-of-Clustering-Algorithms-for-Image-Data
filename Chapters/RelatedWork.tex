\chapter{Related Work}

\todo[inline]{Explain \cite{DBLP:journals/corr/GuptaR15b}.}

\todo[inline]{Explain \cite{DBLP:journals/corr/abs-1711-03091}.}

\todo[inline]{Explain \cite{DBLP:journals/corr/abs-1903-03096}.}

Balcan et al. proposed the two infinite families to interpolate between different linkage strategies \cite{DBLP:journals/corr/BalcanNVW16}, such as shown in equations \ref{eq:algfam1} and \ref{eq:algfam2}.

\begin{equation}
    \begin{aligned}
        \mathcal{A}_1 = \left\{ \left( \min\limits_{u \in A, v \in B} (d(u,v))^\alpha +  \max\limits_{u \in A, v \in B} (d(u,v))^\alpha\ \right)^{1 / \alpha} \middle| \alpha \in \mathbb{R} \cup \{\infty, -\infty\} \right\}
    \end{aligned}
    \label{eq:algfam1}
\end{equation}

Equation \ref{eq:algfam1} shows a distances in the range between single linkage ($\alpha = -\infty$) and complete linkage ($\alpha = \infty$). Balcan et al. also show that $\mathbb{R} \cup \{\infty, -\infty\}$ contains a maximum of $O(n^8)$ different intervals, where each interval $[\alpha_{lo}, \alpha_{hi}]$ represents a different merging behavior.

\begin{equation}
    \begin{aligned}
        \mathcal{A}_2 = \left\{ \left( \frac{1}{\|A\| \|B\|} \sum\limits_{u \in A, v \in B} (d(u,v))^\alpha \right)^{1 / \alpha} \middle| \alpha \in \mathbb{R} \cup \{\infty, -\infty\} \right\}
    \end{aligned}
    \label{eq:algfam2}
\end{equation}

Equation \ref{eq:algfam2} will also result in single linkage for $\alpha = - \infty$ and complete linkage for $\alpha = \infty$. In addition to that, the family $\mathcal{A}_2$ also contains the definition of average linkage ($\alpha = 0$). However, the guarantee for maximum $O(n^8)$ intervals does not apply to this family. A formal guarantee will be $O(n^4 2^n)$, but this thesis will show that the experimental results are much better than the actual formal guarantee.

Balcan et. al also provided a solution to calculate all different merges, however this approach solved the mathematical equations and led to the same clusters being used for a merge quite a often. As our solution only evaluates cases where different clusters get merged, the algorithm described in the following section has a lower runtime as well as a lower lower complexity.
