\chapter{Introduction}


Unsupervised grouping is used in various applications to categorize data observations into similar regions. As an example, similar documents can be combined into clusters so that for a new document or a search query, a list of corresponding documents can be shown \cite{zamir1998web}. The same procedure can also be applied for different tasks such as grouping products \cite{balakrishnan2018product}, searching images \cite{lin2018dimensionality} or detecting anomalies \cite{he2003discovering}. In comparison to supervised learning, the data does not have to be (completely) annotated, i.e.\ potentially expensive labeling work can be avoided by using clustering algorithms.\\

As the amount of available data has been increasing in the past \cite{wamba2015big}, data analysis is more often required for some specific use-case that includes a very specific dataset. State-of-the-art algorithms mostly provide general complexity- and runtime-guarantees, thus worst-case guarantees have to be assumed for the given dataset. However, as large datasets do often not adapt much over time, it is very likely that also runtime and complexity of certain algorithms applied on the given data will not change much. On the other hand, it is not trivial which algorithm can then be used to obtain the optimal results, i.e.\ the optimal clusters of the given data \cite{DBLP:journals/corr/GuptaR15b}.\\

In addition, data is often split into different natural representations. For instance, images on websites can be seen as a matrix of pixels, but visually impaired people would rather use the image's alternative text description. For machine learning experiments it can be difficult to create a model based on various representation as it does not seem to be natural how to stack different data sources such as pixels and alternative texts \cite{cebral2018combining}.\\

This thesis proposes several algorithms to efficiently use a linear combination of clustering algorithms to overcome the hurdle of selecting the proper algorithm for the given data. In addition, the framework this algorithm is built in\footnote{The implementation is published open-source, see \url{https://github.com/manu183/TODO}.} will be also be applied on learning a weighted linear combination of feature representations. The proposed clustering algorithms belong to a specific family that will be introduced in chapter \ref{chapter:background}.